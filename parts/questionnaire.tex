\documentclass[../main]{subfiles}

\begin{document}
\begin{enumerate}
  \item \textbf{Del gráfico obtenido en la primera parte, hallar la velocidad instantánea en el punto $C$.}\\
    Analizando las prolongaciones descritas en \ref{fig:vavg}:
    \begin{align*}
      y_1 &= 0.2363x + 4.328\\
      f_1(0) &= 4.33\\
      y_2 &= -0.2270x + 4.249\\
      f_2(0) &= 4.25\\
      v_i &= \frac{4.33 + 4.25}{2}\\
          &= \qty{4.29}{\cm\per\s}
    \end{align*}
  \item \textbf{¿Qué importancia tiene que las rectas se crucen antes o después del eje de
coordenadas ($\Delta t = 0$)?}\\
    Cuando $\Delta t = 0$ las partículas empiezan su recorrido con la misma velocidad, pero a medida que transcurre el tiempo la velocidad para cada partícula varia. 
    En el caso del tramo $AC$, la velocidad en $x$ es positiva y decreciente, además la aceleración (pendiente) es negativa, es decir, la partícula se mueve en la dirección $+x$ y frena.
    Y en el caso del tramo $BC$, la velocidad en x es positiva y creciente, además la aceleración (pendiente) es positiva, es decir, la partícula se mueve en la dirección $+x$ y acelera.
  \item \textbf{Del gráfico obtenido en la segunda parte, encontrar la aceleración.}\\
    Analizando la recta descrita en \ref{fig:vins}:
    \begin{align*}
      y &= 0.1977x + 0.721\\
      a_i &= y'\\
          &= \qty{0.20}{\cm\per\second\squared}
    \end{align*}
  \item \textbf{Comparar la velocidad instantánea en el punto $C$ de la primera parte con la obtenida al ajustar el gráfico de la velocidad instantánea en función del tiempo de la segunda parte.}\\
    Utilizando el modelo lineal de \ref{fig:vins} para el punto $C$:
    \begin{align*}
      v_i &= 0.721 + 0.1977(19.90)\\
          &= \qty{4.66}{\cm\per\s}
    \end{align*}
    Es observable que hay una diferencia de $\qty{0.37}{\cm\per\s}$, la cual es atribuida a los errores aleatorios en las medidas.
\end{enumerate}
\end{document}
